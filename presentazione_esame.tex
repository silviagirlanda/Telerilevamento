% al 6/02 ore 15
\documentclass{beamer} % indica il tipo di file, in questo caso una presentazione
\usepackage{graphicx} % per inserire immagini
\usepackage{listings} % pacchetto che può essere utilizzato per caricare degli algoritmi direttamente dagli script.
\usepackage{multicol} % per visulizzare gli oggetti su 2 colonne
\usepackage{url} % per inserire pagine web

\definecolor{mygreen}{rgb}{0,0.6,0}

\usetheme{Frankfurt}
\usecolortheme{crane}

\title{\textbf{Analisi vegetazionale della Foresta di Paneveggio (TN)}}
\subtitle {Come Vaia e il bostrico \textit{Ips typographus} hanno modificato il territorio }
\author{Silvia Girlanda}
\date{14 Febbraio 2025}

\begin{document}

\maketitle

\AtBeginSection[] % Do nothing for \section*
{
\begin{frame}{Indice}
\tableofcontents[currentsection]
\end{frame}
}


\section{Background}

\begin{frame}{\textbf{Dove?}}
\begin{figure}
        \centering
        \includegraphics[width=0.5\linewidth]{Area_indagine.jpg}
        \caption{\scriptsize Area d'indagine}
    \end{figure}
    
    \begin{itemize}
        \item Valle del torrente Travignolo
        \item Vicino al Parco Naturale Paneveggio - Pale di San Martino
        \item Prevalenza di abete rosso (\textit{Picea abies}) ma anche: abete bianco (\textit{Abies alba}), larice (\textit{Larix decidua}) e pino cembro (\textit{Pinus cembra})
    
    \end{itemize}
 \end{frame}

\begin{frame}{\textbf{Chi?} (1) La tempesta Vaia: 26 - 30 ottobre 2018 }
14 milioni di alberi abbattuti
\begin{figure}
    \centering
    \includegraphics[width=0.9\linewidth]{Vaia.jpeg}
    \caption{Gli effetti di Vaia}
    \label{fig:enter-label}
\end{figure}
\end{frame}

\begin{frame}{\textbf{Chi?} (2) Il bostrico tipografo}
\begin{figure}
    \centering
    \includegraphics[width=0.40\linewidth]{bostrico.jpeg}
    \caption{{\textit{Ips typographus}}}
    \includegraphics[width=0.40\linewidth]{bostrico_2.jpeg}
    \caption{"Sistema riproduttivo"}
\end{figure}
\end{frame}

\begin{frame}{\textbf{Perchè?} }

L'obbiettivo è vedere come l'area indagata si è modificata nel corso del tempo.
\begin{multicols}{2}
\begin{itemize}
    \item 8 agosto 2017 \begin{large} $\rightarrow$ \end{large} prima di Vaia
    \item 28 giugno 2019 \begin{large} $\rightarrow$ \end{large} dopo Vaia
    \item 31 luglio 2024 \begin{large} $\rightarrow$ \end{large} effetti del  bostrico
\end{itemize}
\columnbreak
                \begin{center}
                \begin{figure}
                    \centering
                    \includegraphics[width=0.9\linewidth]{bostrico-Vaia.jpg}
                \end{figure}
                \end{center}
\end{multicols}
\end{frame}


\section{Codice e risultati}

        \begin{frame}{Organizzazione del lavoro}
            \begin{center}
            Le immagini utilizzate per questo progetto state scaricate dal sito \url{https://browser.dataspace.copernicus.eu} utilizzando i seguenti criteri: \begin{itemize}
                    \item Sentinel - 2
                    \item 92.84 km$^{2}$
                    \item Stagione estiva: 2017, 2019 e 2024
                    \item Copertura di nuvole inferiore al 20\%
                    \item Minor copertura di neve
                    \item Download delle bande 2,3,4 e 8 in formato .TIFF 16 bit
                \end{itemize}
            \end{center}
        \end{frame}

        
        \begin{frame}{In preparazione...}
        \texttt{setwd("C:/Telerilevamento")}
        
        \bigskip
        \bigskip
        Pacchetti utilizzati:
        \bigskip
            \begin{itemize}
                \item    \texttt{library(terra)} %per font Courier
                \item    \texttt{library(imageRy)} 
                \item    \texttt{library(viridis)}
                \item    \texttt{library(ggplot2)} 
                \item    \texttt{library(patchwork)}
            \end{itemize}
        \end{frame}

\begin{frame}{Funzioni utilizzate}
                \begin{multicols}{2}
                    \begin{itemize}
                        \item    \texttt{\textcolor{gray}{setwd()}} 
                        \item    \texttt{rast()} 
                        \item    \texttt{c()}
                        \item    \texttt{par()} 
                        \item    \texttt{im.plotRGB()}
                        \item    \texttt{plot()}
                        \item    \texttt{im.classify()}
                        \item    \texttt{ncell()}
                        \item    \texttt{freq()}
                \columnbreak
                        \item    \texttt{data.frame()}
                        \item    \texttt{View()}
                        \item    \texttt{ggplot()}
                        \item    \texttt{aes()}
                        \item    \texttt{geom_boxplot()}
                        \item    \texttt{ylim()}
                        \item    \texttt{xlab()}
                        \item    \texttt{ylab()}
                        \item    \texttt{ggtitle()}
                    \end{itemize}
                \end{multicols}
 \end{frame}
 
\begin{frame}{Dalle bande all'immagine...}
 \begin{multicols}{2}
    \lstinputlisting[language=R, firstline=35, lastline=51, commentstyle=\color{mygreen}, basicstyle=\footnotesize]{R_code_exam.R}
    \columnbreak
    \begin{figure}
        \raggedleft %per posizionare qualcosa a destra della slide
        \includegraphics[width=0.5\linewidth]{Copernicus_Browser.jpg}
    \end{figure}
    \end{multicols}
\end{frame}

\begin{frame}{True color}
\lstinputlisting[language=R, firstline=55, lastline=58, commentstyle=\color{mygreen}, basicstyle=\footnotesize]{R_code_exam.R}
\begin{figure}
    \centering
    \includegraphics[width=1\linewidth]{par_tc.jpeg}
\end{figure}
        \end{frame}

\begin{frame}{True Color e False Color (nir su banda del rosso)}
\begin{figure}
    \centering
    \includegraphics[width=1\linewidth]{par_tc_nir.jpeg}
\end{figure}
        \end{frame}

 \begin{frame}{NDVI: Normalized Difference Vegetation Index}
            \begin{equation}
                NDVI = \frac{(NIR - RED)}{(NIR + RED)}
            \end{equation}
            \begin{figure}
                \centering
                \includegraphics[width=0.5\linewidth]{NDVI.jpg}
            \end{figure}
        \end{frame}

        
\begin{frame}{NDVI}
\lstinputlisting[language=R, firstline=92, lastline=99, commentstyle=\color{mygreen}, basicstyle=\footnotesize]{R_code_exam.R}
\begin{figure}
    \centering
    \includegraphics[width=1\linewidth]{par_NDVI.jpeg}
\end{figure}
 \end{frame}

\section{Conclusioni}

\section{Fonti e altro}

 \begin{frame}{Fonti utilizzate}
 \begin{multicols}{2}
 \textbf{BIBLIOGRAFIA}
 \begin{itemize}
     \item \textit{Sottocorteccia. Un viaggio tra i boschi che cambiano} di 
Pietro Lacasella e Luigi Torreggiani - People, 2024
 \end{itemize}
\bigskip
 \textbf{SITOGRAFIA}
            \begin{itemize}
                \item \url{https://browser.dataspace.copernicus.eu} \\
                \item \url{https://en.wikipedia.org/wiki/Main_Page} \\
                \item \url{https://parcopan.org/} \\
            \end{itemize}  
\columnbreak
\begin{figure}
    \centering
    \includegraphics[width=0.7\linewidth]{Sottocorteccia.jpeg}
\end{figure}
\end{multicols}
        \end{frame}
        
\begin{frame}{Ringraziamenti}
            \begin{center}
                {\huge Grazie per l'attenzione!}\\
                \bigskip
                \bigskip
                \bigskip
                \bigskip
                \bigskip
                \bigskip
                \bigskip
                \bigskip
            \end{center}
        Il mio profilo GitHub: \url{https://github.com/silviagirlanda} \end{frame}
\end{document}

