\documentclass[12pt]{article} %Spiega che tipo di documento realizziamo
\usepackage{graphicx} % Pacchetto per inserire le immagini
\usepackage{hyperref} %Pacchetto per creare un collegamento ipertestuale
\usepackage{natbib} %Pacchetto per inserire la bibliografia
\usepackage{lineno} %Pacchetto per inserire i numeri delle righe. Line numbers
\linenumbers

%Per ulteriori cose di impaginazione cerca sul web :)

\title{Il mio primo documento LaTeX} %Titolo del documento
\author{Silvia Girlanda}
\date{7 Maggio 2024} % Data del nostro documento, per rimuoverla o tolgo il contenuto tra parentesi o la mantengo come commmento

\begin{document}

\maketitle %Prende tutti i pezzi scritti sopra e li mette all'inizio del documento

%Per fare l'abstract:
\begin{abstract}
    Se partirò a Budapest, ti ricorderai
Dei giorni in tenda, quella moonlight
Fumando fino all'alba, non cambierai
E non cambierò, fottendomi la testa in un night
Soffrire può sembrare un po' fake
Se curi le tue lacrime ad un rave
Maglia bianca, oro sui denti, blue jeans
Non paragonarmi a una bitch così
Non era abbastanza, noi soli sulla Jeep
Ma non sono bravo a rincorrere
\end{abstract}

\bigskip

\textit{Keywords}: Tuta Gold, Budapest, Musica %Per mettere le parole chiave.

\tableofcontents %Funzione per inserire il sommario

\section{Introduction}\label{sec:intro}
\textbf{LA TATUATA BELLA} %Per mettere il testo in grassetto
\textit{Tre Allegri ragazzi morti}  %Per mettere il testo in corsivo
La tatuata bella al lavoro non ci va
Ha messo la camicia ma al lavoro non ci va
Quello che vuole è altro, altro da questa vita
A consumar le dita per gli altri non ci sta
Padrone mostra il muso, vediamo come sei
Padrone mostra il muso, che lingua parlerai?
Da dove arriverai a cercar profitto?
Sta' certo che il mio tempo stavolta non l'avrai
E se lavori duro un motivo ci sarà
E se lavori duro un motivo ci sarà
Che schiavo del lavoro
Che schiavo del lavoro
O solo del pensiero che senza non si può
Ti dico che non vengo, a lavorar non vengo
Starò con la mia bella sul prato a far l'amor
\bigskip %Per creare uno spazio grande tra righe OPPURE doppio backslash \\


La tatuata bella al lavoro non ci va
\smallskip %Per creare uno spazio piccolo tra righe
Ha messo la camicia ma al lavoro non ci va
Quello che vuole è altro, altro da questa vita
A consumar le dita per gli altri non ci sta
Padrone mostra il muso, vediamo come sei
Padrone mostra il muso, che lingua parlerai?
Da dove arriverai a cercar profitto?
Sta' certo che il mio tempo stavolta non l'avrai
E se lavori duro un motivo ci sarà
E se lavori duro un motivo ci sarà
Che schiavo del lavoro
Che schiavo del lavoro
O solo del pensiero che senza non si può
Ti dico che non vengo, a lavorar non vengo
Starò con la mia bella sul prato a far l'amor 
\cite{Prudencio2015} %Per citare la bibliografia 
\citep{Prudencio2015} %citazione tutta tra parentesi
\citet{Prudencio2015} %citazione con testo

Starò con la mia bella sul prato a far l'amor \footnote{Source: Canzone TARM} %Per inserire una nota a piè pagina

Il video della canzone lo trovi su:
\url{https://www.youtube.com/watch?v=0s8LLZtHgyA} %link alla pagina web di riferimento

\section{Methods}
\subsection{Study Area}
\subsection{Algoritms}

%Per scrivere un'EQUAZIONE: 
Prima abbiamo usato la seguente equazione \ref{eq:sum}:
\begin{equation} % Inizio dell'equazione
    T = \sum p_i
\label{eq:sum} %Attribuiamo un'etichetta alla eq. per fare riferimenti
\end{equation} % Fine dell'equazione

In questa tesi abbiamo usato un'equazione \ref{eq:newton}:
\begin{equation}
    F = \sqrt{G \frac{m_1\times m_2}{d^2}} %\sqrt per mettere sotto radice
\label{eq:newton}
\end{equation}

\section{Results}
\section{Discussion}
I risultati ottenuti, come abbiamo visto nella sezione \ref{sec:intro}

In questa tesi abbiamo osservato:
\begin{itemize} %Per fare un elenco puntato:
    \item Lavoro non ci va
    \item Tatuata bella
    \item Altro
\end{itemize}

\begin{enumerate} %Per fare un elenco numerato:
    \item Lavoro non ci va
    \item Tatuata bella
    \item Altro
\end{enumerate}

\newpage %Per aggiungere un'immagine nuova.

%Aggiungiamo un'immagine:
\begin{figure}
    \centering %per centrare la figura nel foglio
    \includegraphics[width=\textwidth]{cha.png} %così l'immagine viene larga quanto il testo. Se volessimo farlo la metà, basterebbe scrivere width=0.5
    \caption{Dipinto di Marc Chagall} %didascalia dell'immagine
    \label{fig:chagall}
\end{figure}

\begin{thebibliography}{999} %è un tipo di bibliografia
\bibitem[Prudencio, 2015]{Prudencio2015} Prudencio, J., et al. "3D attenuation tomography of the volcanic island of Tenerife (Canary Islands)." Surveys in Geophysics 36 (2015): 693-716. %tra graffe c'è il label della citazione; tra quadre la cosa che verrà visualizzata nel testo.

\hline %Disegna una linea orizzontale

\bigskip
HOLA


\end{thebibliography}


\end{document}
