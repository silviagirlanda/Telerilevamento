\documentclass{beamer} % "beamer" indica il tipo di file (presentazione)
\usepackage{graphicx} % Required for inserting images

% Per modificare i colori e i temi della presentazione vai su https://mpetroff.net/files/beamer-theme-matrix/ e cerca il tuo preferito:
\usetheme{Frankfurt}
\usecolortheme{crane}

\title{My first presentation}
\author{Silvia Girlanda}
\date{May 2024}

\begin{document}

\maketitle

%Per mettere il sommario (ricordati di ricompilare più volte per caricarlo):
\AtBeginSection[] % Do nothing for \section*
{
\begin{frame}{Outline}
\tableofcontents[currentsection]
\end{frame}
}

\section{Introduction}

%Le diapositive prendono il nome di "frame"
\begin{frame}{My first slide}
    My first slide
\end{frame}

%Per fare un elenco puntato:
\begin{frame}{Itemization} 
    \begin{itemize}
        \item One
        \item \pause Two % Per fare un'animazione:
        \item \pause Three 
    \end{itemize}
\end{frame}

%Per modificare la dimensione del testo, cerca sul web.
\begin{frame}{Text dimension}
    \scriptsize{Questo è un esempio di frase} %dimensione piccola
    \huge {Questo è un altro esempio di frase} \\ %dimensione più grande; il doppio backslash mi permette di andare a capo 
    Un altro ancora \textbf{esempio}: \textit{tanti esempi!} % Grassetto e corsivo
\end{frame} 

\section{Formulas}
\begin{frame}{Formulas used}
In questa tesi ho utilizzato la formula della \textbf{deviazione standard}:
\smallskip %Per fare uno spazio piccolo
%\bigskip: per fare uno spazio più grande

%Per inserire una formula matematica, cerca come sempre sul web:
\begin{equation}
    \delta = \sqrt{\frac{\displaystyle\sum_{i=1}^N (x_i - \mu)^2}{N}} 
    % Tutte le funzioni usate sopra le abbiamo trovate sul web: \displaystyle mette tutto in modo più fancy.
\end{equation}
    
\end{frame}

\section{Results}
\begin{frame}{Achieved results pt 1}
\begin{figure}
    \centering
    \includegraphics[width=0.5\linewidth]{Frutto_Fiore.jpg} % Solitamente è bene usare 0.9 o anche solo .9 perchè ben si adatta a tutto
    %Rimuovi \caption e \label perchè nelle presentazioni non servono a nulla
    % \raggedright e \raggedleft per spostare a dx o a sx
\end{figure}
    
\end{frame}

%Per mettere due immagini una a lato dell'altra:
\begin{frame}{Achieved results pt 2}
\begin{figure}
    \centering
    \includegraphics[width=0.4\linewidth]{Frutto_Fiore.jpg} 
    \pause \includegraphics[width=0.4\linewidth]{Frutto_Fiore.jpg} %Anche qui con le immagini possiamo mettere \pause per fare un'animazione
\end{figure}
    
\end{frame}

%Per mettere quattro immagini, due sopra e due sotto:
\begin{frame}{Achieved results pt 3}
\begin{figure}
    \centering
    \includegraphics[width=0.3\linewidth]{Frutto_Fiore.jpg} 
    \includegraphics[width=0.3\linewidth]{Frutto_Fiore.jpg} \\
    \includegraphics[width=0.3\linewidth]{Frutto_Fiore.jpg} 
    \includegraphics[width=0.3\linewidth]{Frutto_Fiore.jpg} 
\end{figure}
    
\end{frame}

% Per inserire un testo con font Courier, tipico dei linguaggi di programmazione informatica:
\begin{frame}{Courier}
    In this project we made use of the R package 
    \texttt{imageRy} %funzione per il font Courier
\end{frame}

\end{document}
